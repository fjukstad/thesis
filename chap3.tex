In this chapter we discuss our approach to analyzing high-throughput genomic
datasets through deep analysis pipelines, and its implementation in 
walrus.\cite{walrus} We also evaluate the performance of walrus and show its
usefulness in a precision medicine setting. 

\section{Use Case and Motivation} 
For cancer, high throughput sequencing is the main technology to facilitate
personalized diagnosis and treatment since it enables collecting high quality
genomic data from patients at a low cost. 
Analyzing sequencing datasets require deep analysis pipelines with a large
number of steps that transform raw data into interpretable
results.\cite{diao2015building} These pipelines often consists of in-house or
third party tools and scripts that each transform input files and produce some
output. Although different tools exist, it is necessary to carefully explore
different tools and parameters to choose the most efficient to apply for a
dedicated question.\cite{servant2014bioinformatics} The complexity of the tools
vary from toolkits such as the \gls{gatk} to small custom \emph{bash} or
\emph{R} scripts.  In addition some tools interface with databases whose
versions and content will impact the overall result.\cite{sboner2015primer}

When developing analysis pipelines for use in precision medicine it is necessary
to track pipeline tool versions, their input parameters, and data. Both to
thoroughly document what produced the end results, but also to compare results
from different pipeline runs.  Because of the iterative process of developing
the analysis pipeline, thoroughly investigating emerging patterns and
signatures, it is necessary to use analysis tools that facilitates modifying
pipeline steps and adding new ones with little developer effort. 

We have previously analyzed DNA sequence data from a breast cancer patient's
primary tumor and adjacent normal cells to identify the molecular signature of
the patient's tumor and germline. When the patient later relapsed we analyzed
sequence data from the patient's metastasis to provide an extensive comparison
against the primary and to identify the molecular drivers of the patient's
tumor. 


