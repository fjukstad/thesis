From the discovery of the \gls{dna} structure by Watson and Crick in
1953\cite{watson1953molecular} to
the sequencing of the human genome in 2001
\cite{venter2001sequence,international2001initial} and the massively parallel
sequencing platforms in the later years[6], the scientific advances have been
tremendous. Today, single week-long sequencing runs can produce as much data as
did entire genome centers just years ago.\cite{kahn2011future}  These
technologies allow researchers to collect data faster, cheaper and more
efficient, now making it possible to collect the entire genome from a patient in
less than a days work.

In this chapter we give a background in the different aspects of analyzing and
exploring biological datasets. We highlight the necessary processing steps from
data generation and to interpretation of results. 

\section{High-Throughput Datasets in Research and Medicine} 
\gls{dna} sequencing is the process of determining the order of nucleotides
within a strand of \gls{dna}. 


Precision medicine uses patient-specific molecular information to diagnose and
categorize disease to tailor treatment to improve health
outcome.\cite{national2011toward} Important research goal in precision medicine
are to learn about the variability of the molecular characteristics of
individual tumors, their relationship to outcome, and to improve diagnosis and
therapy.\cite{tannock2016limits} International cancer institutions are therefore
offering dedicated personalized medicine programs, but while the data collection
and analysis technology is emerging, there are still unsolved problems to enable
reproducible analyses in clinical settings. For cancer, high throughput
sequencing is the main technology to facilitate personalized diagnosis and
treatment since it enables collecting high quality genomic data from patients
at a low cost. 

\subsection{Norwegian Women and Cancer} 
The \gls{nowac} systems epidemiology research project is a study designed to
identify the possible relationships between lifestyle and the risk of cancer.
It started its data collection in 1991. In 2006 the study contained
questionnaire information from over 170 000 women. Since then the data
collection started in 1998, the \gls{nowac} postgenome biobank has grown to over
60 000 blood samples and 800 biopsies that have been, or will be, analyzed using
whole-genome gene expression analysis tools. Additionally the biobank contains
information about exposure through questionnaires answered by the participants
of the study.

\section{Preprocessing} 

\section{Analysis Pipelines}
% Watchdog:
% https://bmcbioinformatics.biomedcentral.com/articles/10.1186/s12859-018-2107-4
% maybe this one
% http://gopherdata.io/post/more_go_based_workflow_tools_in_bioinformatics/

\section{Interactive Applications} 

