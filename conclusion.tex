% from zaharia
\section{Lessons Learned}
\section{Broader Impact}
\section{Future Work}
We intend to address few points we aim to address in future work, both in the
MIxT web application as well as the supporting microservices.  The first issue
is to improve the user experience in the MIxT web application.  Since it is
executing many of the analyses on demand, the user interface may seem
unresponsive. We are working on mechanisms that gives the user feedback when the
computations are taking a long time, but also reducing analysis time by
optimizing the underlying R package.  The database service provides a sufficient
interface for the MIxT web application. While we have developed the software
packages for interfacing with more databases, these haven't been included in the
database service yet. In future versions we aim to make the database service an
interface for all our applications.  We also aim to improve how we capture data
provenance. We aim to provide database versions and meta-data about when a
specific item was retrieved from the database.  One large concern that we
haven't addressed in this paper is security. In particular one security concern
that we aim to address in Kvik is the restrictions on the execution of code in
the compute service. We aim to address this in the next version of the compute
service, using methods such as
AppArmor\footnote{\url{wiki.ubuntu.com/AppArmor}.} that can restrict a program's
resource access. In addition to code security we will address data access,
specifically put constraints on who can access data from the compute service.
We also aim to explore different alternatives for scaling up the compute
service.  Since we already interface with R we can use the
Sparklyr\footnote{\url{spark.rstudio.com}.} or
SparkR\footnote{\url{spark.apache.org/docs/latest/sparkr.html}.} packages to run
analyses on top of Spark.\cite{zaharia2010spark} Using Spark as an execution
engine for data analyses will enable applications to explore even larger
datasets.
