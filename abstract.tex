\begin{abstract}
    There is a rapid growth in the number of available biological datasets due
    to the decreasing cost of collecting and storing the data, together with
    high-throughput data collection instruments. Modern instruments enables
    analysis of biological data at different levels, from DNA sequences
    through cell structures and up to the function of organs. This leads to the
    potential for novel insights to the underlying biological mechanisms in the
    development and progression of diseases such as cancer. The wide range of
    these different biological datasets has led to the development of a wealth
    of software packages and systems for researchers to use in data exploration
    and analysis.

    Researchers tailor the exploration and analysis of their datasets using a
    range of different tools and systems. Because of the specialized nature of
    each of the analyses, few of the tools provide useful interfaces for
    analyses across programming languages and frameworks. In addition, reporting
    the details of an analysis becomes a tedious task because of the long list
    of tools. The lack of such details complicates the process of reproducing
    results and reusing methods or tools for other datasets. This increases both
    analysis time and leaves unrealized potential for scientific insights.

    This dissertation argues that, instead, we can design systems for
    exploring and analyzing high-throughput biological datasets from small
    composable pieces. 
    
    % In particular we show the viability of software container
    % technologies together with the
    % to build applicationsIn particular we show how 


    % This dissertation argues that, instead, we can design a unified approach
    % that integrates disparate systems and data into fully reproducible
    % biological data analysis and exploration frameworks. In particular, we show
    % how software container technologies together with well-defined interfaces,
    % configurations, and orchestration provide the necessary foundation to build
    % reproducible analysis pipelines for biological datasets, as well as highly
    % interactive data exploration applications. 

    We show the feasibility of our approach through a number of different
    applications for analyzing and exploring biological datasets.  We developed
    these applications to meet the requirements of researchers in systems
    epidemiology and precision medicine. We evaluate the approach through these
    systems using real datasets and analyses. Our results show that our approach
    can be used to enable reproducible data analysis and exploration of
    high-throughput biological datasets while still providing the performance of
    related systems.
    
\end{abstract}

